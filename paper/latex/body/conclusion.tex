% !Mode:: "TeX:UTF-8" 

\BiAppendixChapter{结\quad 论}{Conclusions}

文本立场分析通过判断文本对主题目标的倾向,分析作者对主题目标所持有的立场。本课题在分析了文本立场分析在国内外的研究现状后,发现现有研究还存在以下问题:(1)以往有关文本立场分析的研究主要集中在议会辩论与网上论坛辩论等较为规范的文本,对微博和Twitter等表现形式更自由与多样化的社交媒体文本的研究较少。(2)现有对文本立场分析的研究途径主要集中在基于传统语义特征的机器学习,而这些语义特征主要来源领域专家的手工构造,需要大量的人力资源,模型迁移到其他领域的成本也高。(3)现有的文本立场分析方法没有考虑主题目标信息,而将文本立场分析看成简单的文本情感分析任务。为解决上述问题,本课题的主要工作包括以下两点:

(1)受条件编码在文本蕴含任务上性能突破的启发,将条件编码模型引入到文本立场分析任务。为验证引入主题目标信息是否能提高文本立场分析性能,设计了对比实验,实验结果表明以条件编码的方式引入主题目标信息能显著提高文本立场分析的性能。结合社交文本立场分析的特点,改进了条件编码模型,在SemEval2016英文Twitter数据集和NLPCC2016中文为微博数据的微平均F1值分别为0.671与0.698,在不需要外部收集语料与手工设计特征的前提下,条件编码模型性能已经接近评测最佳系统。

(2)基于主题目标对文本信息内容有着不同的侧重点为基础,本文将主题目标信息作为注意力机制的导向,给予文本信息不同权重的关注度,在授予不同权重文本信息中挖掘有关立场分析的模式。由于注意力机制与条件编码是从两种不同角度引入主题目标信息,本文提出一种融合注意力机制与条件编码的文本立场分析方法。通过设计对比实验,表明条件编码与注意力机制方式在不同主题目标任务下各有优劣,融合后模型在SemEval2016数据集和NLPCC数据的微平均F1值分别为0.689与0.716,对比两个数据集评测任务的最优系统分别提高了1.08\%和0.61\%,表明融合注意力机制与条件编码模型在社交媒体文本立场分析任务上的有效性。

本研究针对现有问题提出一些解决的方案,但是由于时间限制,还存在若干缺点和不足。例如,未将深度学习和传统特征机器学习方法结合,形成一个更加强健的解决方案;此外,无监督或弱监督的立场分析方法也是将来值得尝试的方向。