% !Mode:: "TeX:UTF-8" 

\BiAppendixChapter{结\quad 论}{Conclusions}

文本立场分析是挖掘文本作者对主题目标所持有的立场的技术。本课题在分析了文本立场分析在国内外的研究现状后,发现现有研究还存在以下问题:(1)现有对文本立场分析的研究途径主要集中在基于传统语义特征的机器学习,构建模型成本较高。(2)现有的文本立场分析方法通常没有考虑主题目标信息,而将文本立场分析看成简单的文本分类任务。为解决上述问题,本课题的主要工作包括以下两点:

(1)针对已有立场分析研究未引入主题目标信息的缺点,将条件编码模型引入到文本立场分析任务。使主题目标信息以一种“先验知识”的方式参与对文本的编码。根据社交立场文本的特点,本文提出分别用单向LSTM与双向LSTM编码主题目标信息与文本信息。实验结果表明以改进后条件编码的方式引入主题目标信息能显著提高文本立场分析的性能。改进后模型在SemEval2016英文立场分析数据集和NLPCC2016中文立场分析数据集的微平均F1值分别为0.671与0.698,在不需要外部收集语料与手工设计特征的前提下,所提出的方法具有较好的性能。

(2)基于主题目标对文本信息内容有着不同的侧重点为基础,本文将主题目标信息作为注意力机制的导向,给予文本信息不同权重的关注度,在不同权重文本信息中挖掘有关立场分析的模式。由于条件编码与注意力机制分别从“编码”与“解码”两个不同角度引入主题目标信息,本文提出一种融合注意力机制与条件编码的文本立场分析方法。实验表明条件编码与注意力机制方式在不同主题目标任务下各有优劣,融合后的模型具有显著的提升。融合后模型在SemEval2016英文立场分析数据集和NLPCC2016中文立场分析数据集的微平均F1值分别为0.689与0.716。对比两个数据集评测任务的最优系统均有提升,表明本文提出融合注意力机制与条件编码模型在社交媒体文本立场分析任务上的有效性。

本研究针对现有问题提出一些解决的方案,但是由于时间限制,还存在若干缺点和不足。例如,未将深度学习和传统特征机器学习方法结合,形成一个更加强健的解决方案;此外,无监督或弱监督的立场分析方法也是将来值得尝试的方向。