% !Mode:: ,TeX:UTF-8" 

\BiChapter{绪论}{Methods of inserting figures}


\BiSection{课题来源}{Figures inserting standard from graduate school}
本课题来源于国家自然科学基金重点项目《社交媒体中文本情感语义计算理论和方法》、国家自然科学基金项目《文本情绪计算框架、模型和方法研究》、广东省数据科学工程技术研究中心开放课题《社会化媒体大数据群体情感深层理解与预测研究》、《深圳市孔雀计划技术创新项目结合脑科学和深度学习的文本情绪计算研究及其在社会群体情绪分析的应用》。



\BiSection{课题研究的背景、目的与意义}{Captions and descriptions of figures}
例如政府想确定公民对新政策的意见和态度,可能会使用在其新政策相关的微博、论坛中的帖子来收集反馈。现阶段研究方式主要基于人工调查分析,但人工调查有成本高、时效性差等缺点。在这一时代背景下,如何利用现阶段比较成熟的自然语言处理、机器学习、深度学习等技术分析出用户的立场成为了一个待解决的问题。

文本情感分析,指用自然语言处理、文本挖掘以及计算机语言学等方法来识别和提取原素材中的主观信息。通常来说,情感分析的目的是为了找出说话者/作者在某些话题上或者针对一个文本的观点的态度。这个态度或许是他或她的个人判断或是评估,也许是他当时的情感状态,或是作者有意向的情感交流。文本情感分析的一个基本步骤就是将文本中的某段已知文字的两极性进行分类,这个分类可能是在句子级、功能级。分类的作用就是判断出此文字中表述的观点是积极的、消极的、还是中性的情绪。更高级的“超出两极性”的情感分析还会寻找更复杂的情绪状态,比如“生气”、“悲伤”、“快乐” 等等。

目前,文本情感分析领域的主要方法为基于规则和基于机器学习的方法。前者借助已有情感词典和语言学基础,利用情感词汇单元的极性以及其他语言成分对情感词汇结合、强化、传播等作用,达到文本的情感分类的目的。这种方法虽然无需标注大量的训练数据,但对情感词典资源和语言学基础有比较高的要求。大量情感新词的不断出现,很难全面的归纳出一套完整精确的情感字典,而且网络社交平台口语化和特性化的表达也给语言规定带来了巨大的挑战,因此文本情感分析的主流集中在基于机器学习的方法。基于机器学习的方法通过特征工程的方法选择特征词,然后根据特征词构造出文本的特征表示,再结合已经标注的训练样本构造情感分析模型。这一方法为文本情感分析的表现带来了显著的提升。但基于机器学习的情感分析方法却面临表示向量稀疏和特征词构造复杂等问题。近年来,深度学习在自然语言处理领域取得广泛的进展.借助与Word2Vec\cite{mikolov2013efficient}、循环神经网络等深度学习技术可以将稀疏词表示变为稠密、连续、低维的向量,并利用端到端分类减少特征构建的复杂性。基于深度学习的情感分析方法已经取得较高的性能。

文本情感分析将文本分为正向、负向、中性等多种类型,但是该分类方式经常无法满足很多场景的需求。近年来,在给出特定话题或目标的前提下,分析文本对该目标所持支持、反对、中性立场的立场分析逐渐得到关注,并出现了一些针对网络论坛辩论、以及微博和Twitter语料的立场分析研究与评测。表面上立场分析与传统的情感分析任务具有较高的相似性,实际上两者存在显著的不同。情感分析只关注文本本身表达的正向、负向的情感,无需关注其他的内容。而立场分析需要进一步分析出文本对特定的目标的立场,这种立场意图有可能是显式或者隐式的,这也极大增加的研究文本立场分析的难度。例如有关美国大选“希拉里”为主题,文本内容为“希拉里是个十足的病态的骗子”,此文本明显归为负向的情感,对于“希拉里”主题持“反对”态度, 但是假设以“希拉里”的竞争对手“特朗普”为主题,文本的情感分析依旧没有改变,但是由于基于不同的立场改变,文本的立场也改变成“支持”了。在文本立场分析中,同一个立场,其文本表达可以是正向或者负向的。文本立场分析与文本情感分析有着显著的区别,本文的主要研究内容和组织结构是在文本情绪分析上的更进一步的深入研究。

本文以社交媒体文本的立场分析作为研究目的,研究基于现有深度学习框架并结合立场评价对象的信息分析社交媒体立场的方法,此方法结合了注意力机制和卷积神经网络,改善现有研究未充分挖掘立场主题和立场文本的关系,从而提高文本分析任务上的性能。该方法对文本立场分析研究有一定的科学贡献;同时,本文研究的立场分析方法在商业智能、舆情分析、政府决策中具有较高的应用价值。



\BiSection{国内外相关研究概况}{Recommended figure format applied in \LaTeX}
文本情感分析和立场分析作为自然语言处理领域的热点问题,吸引很多的研究者的关注,研究者们也取得可观的成果。本文将从文本情感分析研究现状、文本立场分析研究现状方面对国内外相关研究展开介绍。


\BiSubsection{文本情感分析研究现状}{Recommended figure format applied in \LaTeX}

文本情感分类可以按照处理粒度分为词语级、短语级、句子级、篇章级情感分类任务。目前,大部分文本情感分类研究可分为两种主要研究思路:基于情感词典/规则和知识库的方法以及基于机器学习的特征分类的方法。前者主要是依靠人工构建的情感词典或领域词典,以及主观文本中带有情感极性的组合单元,来获取情感文本的极性。后者主要是使用机器学习的方法,选取有效的分类特征构造分类器来完成分类任务。近年来,基于深度神经网络和表示学习的方法在情感分类领域得到广泛重视。因此,本节在回顾文本情感分类研究现状时,将对基于表示学习的方法进行单独讨论。

基于情感词典、规则和常识库是文本情感分类的基本方法。基本思想是在待分类文本中对情感词典记录的词语、规则和知识库条目进行匹配,利用匹配条目中记录的情感分类信息生成文本的情感分类结果。典型的工作如Godbole\cite{godbole2007large}构建了基于英文词典的情感分析系统,并用该系统来评估新闻文本中各个实体(人物、地点、时间)的情感倾向。 毛峡等\cite{毛峡2011人机情感交互}以OCC认知情感模型为基础,结合知网常识库进行扩展,制订了 22 种情感识别规则,对应识别 OCC模型定义的22种情感类型。Balahur\cite{balahur2011detecting}建立一个常识库用来记录各种引发情感的事件,从而识别出没有明确情感指示词汇出现的文本中的情感类别。其核心思想是把情景以本体的形式建模成一连串动作及与其相应的情感。Taboada\cite{taboada2011lexicon}提出使用情感词典中情感词以及程度副词和否定词特征对句子或短语的情感分类。Agarwal\cite{agarwal2015sentiment}基于ConceptNet常识库提取文本表达中的重要概念以及相关特征。随后利用SenticNet、SentiWordNet和 General Inquirer构建上下文情感词典,依据各个特征的重要性对文本进行情感分类。Tromp\cite{tromp2015pattern}结合Plutchick的情感轮模型和基于规则的方法,构建了一种社会化媒体中情感分类的框架,提高了文本情感分类准确度。总体来看,基于情感词典/规则和知识库的情感分类准确率较高,但由于情感词典和常识库规模的限制,覆盖率较低。同时此类方法对分词、词性标注、规则匹配等的准确性要求较高,系统内部错误传递影响较大。[30]

机器学习近年来基于机器学习的情感分类研究得到快速发展。Aman应用情感词的 Unigram 特征,结合支持向量机(Support Vector Machines, SVMs)进行文本情感分类。Abbasi\cite{abbasi2008affect}在多种不同的数据集上选取多种不同的特征,采用集成分类器方法,实现对文本的情感分类。Keshtkar\cite{keshtkar2012hierarchical}提出分层分类方法实现博客作者的心情(Mood)分类。该方法由粗到细,将100多种心情类别分成五个层次。通过分层训练 SVM 分类器,先对大类进行分类,而后再进行下一级小类的分类。Mohammad\cite{mohammad2012portable}将情感词典与 n-gram 特征结合,使用 SVM分类器进行情感分类。随着机器学习技术的发展,研究者提出了多种分类算法和特征集合的方法。[Xia et al. 2011]使用词性、词的关系两大类特征集,朴素贝叶斯、最大熵和SVM等三种基分类器,固定组合、权重组合和meta-classifier三种分类器集成策略,通过特征、分类器、集成策略组合,有效提升文本情感分类性能。针对文本中跨句子的复杂语言结构,Yang提出了一种融合局部和全局层次的情感分类方法。在学习过程中把语言信息编码为软约束,并通过后正规化的方法将软约束转化为条件随机场(Conditional Random Fields,CRFs)的特征参数,应用条件随机场解决情感分析问题。[29][张林等,2014]提出了一种基于短评论特征共现的特征筛选方法,将短小评论中的优势信息和传统的特征筛选方法相结合,在筛选掉无用噪音的同时增补有利于分类的有效特征,有效提高短文本的情感分类效果。[张志琳等, 2015]提出选取词汇化主题特征、情感词内容特征和概率化的情感词倾向性特征用于微博情感分类。此外,基于跨领域、跨语言迁移学习的情感分析也得到了关注。张博\cite{张博2015一种基于跨领域典型相关性分析的迁移学习方法}将典型相关性分析引入情感分析迁移学习,基于特征映射迁移学习的思路,在保持各领域特有特征与领域共享特征相关性的基础上选择合适的基向量组合训练分类器,使降维后的相关特征在领域间具有相似的判别性,有效提高跨领域情感分类准确率。Gui\cite{gui2015novel}应用类噪音估计算法,对跨语言迁移学习中负面样本进行检测和过滤,提高了在目标语言情感分析的性能。

随着基于深度学习的分布式表示算法的提出,能够利用神经网络获得对应的词语复合、句子复合、段落复合的分布式表示,在分布式表示学习的基础上进行文本情感分析的方法正在成为最近的研究热点。Socher\cite{socher2013recursive}利用递归神经张量网络模型对Stanford情感树库中标注了句子分析树和节点情感标签的句子进行复合表示学习。并应用该模型对输入句子进行逐个复合节点的情感判断,最终在根节点上得到整个句子的情感。 Kim\cite{kim2014convolutional}将卷积神经网络进行部分改进后用于文本分析任务。实验结果证实了卷积神经网络在文本分析任务上的有效性。在这个的工作之后。梁军\cite{梁2015基于极性转移和}将长短时记忆模型(Long Short Term Memory, LSTM)模型扩展到基于树结构的递归神经网络(RNN)上,用于捕获文本更深层次的语义语法信息,根据句子前后词语间的关联性引入情感极性转移模型,到达了较好的情感分类性能。在文档级别的情感分类任务中,Tang\cite{tang2015deep}提出了一种自底向上学习文档级向量表示的模型。该模型首先使用卷积神经网络(CNN)或长短时记忆网络(LSTM)学习句子级向量表示,随后,利用gated RNN将句子语义和联系自适应编码进文档的向量表示中。利用不同级别的文本的复合表示学习,提高情感分类的性能。针对文本情感分类任务通常面对的样例数据不平衡的问题,Xu\cite{xu2015show}提出了一种利用词向量表示构造平衡的训练数据集的方法。在依次构造词向量和句子向量表示后,使用SMOTE算法生成少数类别的样例,并最终构造出均衡的训练数据,有效提高了文本情感分类的性能。

社交媒体文本立场分析
\BiSubsection{社交媒体文本情感分析研究现状}{Recommended figure format applied in \LaTeX}
文本情感分析在社交媒体应用主要集中在在线新闻评论的观点挖掘和微博(Twitter)文本情感分析。Potthast\cite{potthast2012information}指出有关在线新闻评论的研究工作主要集中在信息检索方面,例如过滤,排序和评论摘要。对比于上述信息检索相关技术,在新评论的观点挖掘研究方向上探索相对较少。随着微博的风靡,与之相关的研究得到学术界和工商界的广泛关注。中文微博情感分析作为微博分析的重要基础任务,吸引了很多研究者的关注。以下分别介绍文本情感分析在两者的应用与研究。

对新闻评论的情感分析主要集中在极性检测和情感检测上。现有的研究大部分使用有监督的学习方法。Zhou\cite{zhou2010research}对比了不同的特征在情感分析上的作用,Chardon\cite{chardon2013measuring}探讨了使用话语结构预测新闻反应的作用。 Zhang\cite{zhang2016ecnu}提出了一种用于标记情绪(如悲伤,惊喜和愤怒)的元分类器。在他们的方法中,作者在评论中使用了两个异构信息源:基于内容的信息和情感标签。Jakic提出了一种自动预测新闻反应中情绪极性的方法。在这项工作中,作者使用了领域基础知识和迁移学习,得到了Twitter数据迁移训练的分类器。moreo\cite{moreo2012lexicon}提出了一种基于词法的方法,可以适应不同的领域。在他们的工作中,他们使用WordNet关系设计构建了一个结构化的词典。 Zhao\cite{zhao2010jointly}提出一种利用评论的聚类对评论分类的无监督的方法,检索每一个评论的关键句子并提取其中的命名实体作为评论的目标。

从读者调查和最近共享的比赛任务看,Twitter的情感分析研究很多。Twitter情感分析比起正式本文具有更大的挑战。它通常短且不正式,包含很多特殊的标记标签和表情符号和俚语,字母大小写也不一致。另一个问题是它倾向于不遵守语法规则,包含很多错误的拼写和很多单词的缩写。因此之前一些研究提出的了针对Twitter本文情感分析的方法。Kiritchenko\cite{kiritchenko2014sentiment}提出根据Twitter文本短且非正事的特点,调整一系列表层特征提取方式,比如存在/没有积极和消极的表情符号、标签、大写的字母和重复了字母的单词(比如sweeettt)。最近几年,在分析对选举人Twitter政治情感、情绪和目的上研究热情很高Mohammad\cite{mohammad2015sentiment}研究了如何确定政治联盟。


\BiSubsection{文本立场分析研究现状}{Recommended figure format applied in \LaTeX}
上文论述了文本立场分析与文本情感分析有着本质的区别,文本立场分析更加关注文本反应出作者对于某一特定目标主题所持的立场和倾向。立场分析需要结合目标主题和情感信息,这比单独考虑文本的消息更加具有有挑战,对模型的建模能力也有更高的要求,现有立场分析的研究主要集中在国会辩论和网上辩论,这些领域的辩论作者会提出或者表现出清晰的立场倾向,语法和论述结构也相对固定化。但是对于一些用户主导的且表达方式更加随意内容,例如微博、Twitter、商品评论的立场分析的研究也相对较少。

现有立场分析的研究的主要基于有监督的学习,在Somasundaran\cite{somasundaran2009recognizing}研究中,建立了论点触发词典,词典用来定位与抽取不同的论点,这些提取的论点、情绪表达、以及其目标作为立场分析分类器的特征。作者的实验表明单独用词频做特征比外加其他句法结构依赖性能效果并不会差很多,说明了在作者的任务上,其他特征对立场分析的影响相对较少。Hasan\cite{hasan2013stance}在Anand使用的特征集的基础上,使用条件随机场(Conditional Random Fields,CRF)

标注用户交互序列特征、整数线性规划(integer linear programming,ILP)建模作者意识形态约束特征,在 SVMs 分类器上的表现取得了最高 10\%的提升。通过两个支持立场语言的集合,Faulkner\cite{faulkner2014automated}研究了文档级别在学生的文章中立场分析。Hasan\cite{hasan2013stance}等提出了连续的评论之间不是完全独立的假设,连续的评论可以把问题定义成一个序列标注问题。Ahmed\cite{ahmed2010staying}等提出一个该版本的主题模型算法(Latent Dirichlet Allocation, LDA),作者把每一个词看成是立场倾向和主题的相互结合。Tutek使用随机森林(Random Forest, RF)、迭代决策树 ( Gradient Boosting Decision Tree, GBDT )、 逻辑斯蒂回归 ( Logistic Regression, LR)和 支持向量机(Support Vector Machine, SVM) 四种基础的分类模型,构造多组语言学特征, 作者利用了集成学习的思路,以F-measure作为集成学习的优化目标,线性组合四种基础模型的分类概率。其实验证明了基于多种基础分类的输出线性组合能明显的提高立场分类的效果。Sobhani的研究表明,一段文本表达的立场和文本的情感分析存在一定的关联 。若将文本的情感作为文本立场分析的特征时,能显著的提升文本立场分析的性能。Rajadesingan\cite{rajadesingan2014identifying}研究用户级别的立场分析,作者提出了如果多个Twitter用户转发同一对有争议的话题,那这些用户很大可能拥有同意的立场。

在自然语言处理领域中,传统的基于特征工程的方法能取得较好的效果,然而对文本提取特征词的特征工程需要大量的人力劳动和先验知识。而且基于传统的词向量的表达方式有语言缺乏关联,高纬特征稀疏特征表示,维度灾难等缺陷。Mikolov\cite{mikolov2013efficient}提出了Word2Vec模型,解决了词向量训练速度慢,效率低的缺点。其利用了CBOW和Skip-Gram的两种语言模型,且创新性的提出了Hierarchical Softmax和负采样的词向量加速方法。为后续深度学习模型能在自然语言处理任务上打下了夯实的基础。有关文本立场分析的研究也开始关注能自动提取特征的深度学习。Wan(2016)等利用多卷积核文本CNN的模型对Twitter文本进行有监督的立场分析,作者利用词窗口大小3,4,5的卷积提取文本中的特征,这算方法借鉴了自然语言处理中的N元组词(N-gram)的思想。此模型在Semeval2016-TaskA取得较好的成绩。Zarrella\cite{zarrella2016mitre}使用迁移学习的方法,首先从大量的不标注Twitter数据中,选取了词频高于100的词汇,然后用Word2vec模型预训练好了每一个词的256维度的词向量,后面通过词向量相似度选取比较关键的以\#为前缀主题标签,通过训练神经网络预测主题标签。后通过迁移学习的思想对网络结构进行微调来达到立场分析的目的。实验结果表明,外部无标注数据的使用可以给有监督学习提供一些帮助,提高有监督学习的性能。

上述主要叙述了基于有监督的立场分析方法,虽然有监督的方式可以准确拟合训练集中,构造出对训练集有显著效果的特征, 但是标注大量有类标的训练集需要大量的人工成本,模型泛化能力也相对较弱,技术的应用场景也十分有限。为解决上述的缺点,研究人员开始展开对文本立场立场分析的无监督学习和弱监督学习的研究。 Johnson\cite{johnson2016all}等通过基于不同方面的特征,构建6个局部弱监督基分类器。基于这些弱监督分类器在概率软逻辑(Probabilistic Soft Logic,PSL)算法的结合下,组成一个全局的弱监督分类模型。此模型在有关美国的32名政治人物的Twitter文本的立场分析任务中取得较好的性能。Augenstein\cite{augenstein2016usfd}使用基于词袋的自动编码机(Auto-Encoder)学习文本的特征表示,并将学习得到的特征用于有监督的分类器中,解决了训练数据缺失的问题。



\BiSection{本文的主要研究内容和组织结构}{Recommended figure format applied in \LaTeX}

本文主要研究基于融合主题目标信息的深度学习对社交媒体中的文本进行立场分析

本文的工作主要从以下两个方面展开:

(1) 基于条件编码LSTM模型在社交媒体文本中立场分析,受条件编码在文本蕴含任务上性能突破的启发,将条件编码模型引入到文本立场分析任务。设计了只用LSTM模型编码文本信息、两个独立LSTM分别编码主题目标信息与文本信息和以主题目标信息为条件编码文本信息的对比实验。实验结果表明以条件编码的方式引入主题目标信息能显著提高文本立场分析的效果。并结合文本立场分析的特点,改进了条件编码模型。通过在SemEval2016英文立场数据集和NLPCC2016中文立场数据集对比其他模型性能,表明在改版后的条件编码LSTM模型在社交媒体文本立场分析具有较好的性能。

(2) 基于注意力机制的卷积神经网络的社交媒体文本立场分析,受注意力机制在图像处理、语音识别、机器翻译上的性能的突破,将注意力机制应用于文本立场分析任务中。基于不同主题目标对文本信息内容有着不同的侧重点为基础。本文将文本立场分析中主题目标作为注意力导向,给予文本信息不同权重的关注度,其后利用卷积神经网络挖掘已经受于不同关注度文本信息中有关立场分析的模式。模型在SemEval2016英文数据集上达到0.680的微平均F1值,在NLPCC2016中文数据集上达到了0.716的微平均F1值。对比条件编码LSTM模型分别提升了0.81\%和1.76\%,对比两个数据集评测任务的最优系统分别提高了0.20\%和0.61\%。实验结果表明了基于注意力机制的卷积神经网络在社交媒体文本立场分析的有效性。

本文第1章为绪论,其余各章主要内容如下:

第2章介绍立场分析领域的常见技术,分别从文本情感分析相关研究、文本立场分析相关研究和基于深度学习的立场分析研究等进行分析和回顾。

第3章介绍基于条件编码LSTM模型的神经文本立场分析。本章利用了主题目标信息作为条件来编码文本信息的LSTM模型来解决文本立场分析任务。在此章首先介绍条件编码模型,在实验部分简介了词向量预训练、数据集、评测指标和数据预处理,后介绍实验结果与分析。

第 4 章介绍基于注意力机制的卷积神经网络的社交文本立场分析。此模型将文本立场分析中主题目标作为注意力导向,给予文本信息不同权重的关注度,其后利用卷积神经网络挖掘其中有文本关立场分析的模式。随后介绍了对该方法的验证实验和分析。

最后,结论回顾本文的主要研究内容,归纳并总结研究的不足支持和待解决的问题。

