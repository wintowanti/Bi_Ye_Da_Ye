% !Mode:: ,TeX:UTF-8" 

\BiChapter{绪~~~论}{Methods of inserting figures}


\BiSection{课题来源}{Figures inserting standard from graduate school}
本课题来源于国家自然科学基金重点项目《社交媒体中文本情感语义计算理论和方法》。


\BiSection{课题研究的背景、目的与意义}{Captions and descriptions of figures}
随着互联网与移动互联网的飞速发展和多样化网络交流软件的普及使用,越来越多的网络用户可随时随地浏览热点新闻报道,与此同时借助微博、Twitter、知乎等平台上围绕各种新闻报道、热点社会事件等话题发表自己的观点和表达自己的立场与情绪。这些海量的评论数据具有重要的研究价值。例如政府想确定公民对新政策的意见和态度,会使用在其新政策相关的微博、论坛中的帖子来收集反馈信息。现阶段研究方式主要采取人工调查分析,但人工调查有成本高、时效性差等缺点。在这一时代背景下,如何利用现阶段比较成熟的自然语言处理、机器学习、深度学习等技术分析出用户的立场成为了一个待解决的问题。

文本情感分析,指应用自然语言处理、文本分析、计算机语言学等技术系统的定位、提取、量化文本信息中作者所有表达的个人主观信息。文本情感分析的一个基本步骤就是将文本中的某段已知文字的两极性进行分类,这个分类可能是在短语级、句子级和文档级。文本情感分析就是判断出此文字中表述的观点是积极的、消极的或者中性的情绪。在文本情感分析后更深入的研究的是文本情绪分析,文本情绪分析不单单把文本分成上述三个极性,而是分类成更加复杂的情绪状态,比如“快乐”、“悲伤”、“愤怒”和“厌恶”等。

目前,文本情感分析研究方向主要包含基于规则与词典的方法和基于特征构建的机器学习的方法。前者借助已有情感词典和语言学基础,利用情感词汇单元的极性以及其他语言成分对情感词汇进行结合、强化、传播等操作,达到文本的情感分类的目的。这种方法虽然无需标注大量的训练数据,但对情感词典资源和语言学基础有比较高的要求。大量情感新词的不断出现,很难全面的归纳出一套完整精确的情感字典,而且网络社交平台口语化和特性化的表达也给语言规则的建立带来了巨大的挑战,因此研究人员主要采用基于机器学习的方法来对文本情感进行分析。基于机器学习的方法首先需要应用特征工程来抽取和归纳相应特征词,然后根据特征词构造出文本的特征表示,再结合已经标注的训练样本构造情感分析模型。这一方法为文本情感分析的表现带来了显著的提升。但基于机器学习的情感分析方法却面临表示向量稀疏和特征词构造复杂等问题。近年来,借助深度学习技术,自然语言处理领域获得了快速的推进。Word2Vec\citeup{mikolov2013efficient}、循环神经网络等深度学习技术可以将稀疏词表示变为稠密、连续、低维的向量,并利用端到端分类减少特征构建的复杂性。基于深度学习的情感分析方法已经取得较高的性能。

文本情感分析将文本分为正向、负向、中性等多种类型,但是该分类方式经常无法满足很多场景的需求。近年来,在给出特定话题或目标的前提下,分析文本对该目标所持支持、反对、中性立场的立场分析逐渐得到关注,并出现了一些针对网络论坛辩论、以及微博和Twitter语料的立场分析研究与评测。表面上立场分析与传统的情感分析任务具有较高的相似性,实际上两者存在显著的不同。情感分析只关注文本本身表达的正向、负向的情感,无需关注其他的内容。而立场分析需要进一步分析出文本对特定的目标的立场,这种立场意图有可能是显式或者隐式的,这也极大增加的研究文本立场分析的难度。例如有关美国大选“希拉里”为主题,文本内容为“希拉里是个十足的病态的骗子”,此文本明显归为负向的情感,对于“希拉里”主题持“反对”态度;但是假设以“希拉里”的竞争对手“特朗普”为主题,文本的情感分析依旧没有改变,但由于主题目标的改变,文本的立场也改变成“支持”了。在文本立场分析中,同一个立场,其文本表达可以是正向或者负向的。文本立场分析与文本情感分析有着显著的区别,本文的主要研究内容和组织结构是在文本情绪分析上的更进一步的深入研究。

本文以社交媒体文本的立场分析作为研究目的,研究基于现有深度学习框架并结合立场分析中的主题目标信息分析社交媒体立场的方法,此方法应用了条件编码和注意力机制,改善现有研究未充分挖掘立场主题目标和立场文本的关系,从而提高文本分析任务上的性能。该方法对文本立场分析研究有一定的科学贡献,同时,本文研究的立场分析方法在商业智能、舆情分析和政府决策中具有较高的应用价值。



\BiSection{国内外相关研究概况}{Recommended figure format applied in \LaTeX}
作为自然语言处理领域中的热点问题,文本情感分析和文本立场分析,吸引很多的研究者的关注,研究者们也取得可观的成果。本文将从文本情感分析研究现状和文本立场分析研究现状两个方面对国内外相关研究展开介绍。


\BiSubsection{文本情感分析研究现状}{Recommended figure format applied in \LaTeX}

目前,文本情感分析任务解决途径主要包含基于规则与词典的方法和基于特征工程机器学习的方法。前者主要是依靠人工构建的情感词典与相关领域专家制定匹配规则,计算文本与情感字典或规则匹配程度,从而得到文本的情感极性;后者首先构造一些对情感分析有效的文本特征,然后通过一些主流的机器学习方法在训练集的文本特征中抽取情感分类的模式。以下分布介绍这两种方法,在结尾介绍情感分析在社交媒体文本中有关研究。

基于情感词典、规则和常识库是文本情感分析的早期方法。在文本中匹配情感词典中规则和短语,利用匹配条目中记录的情感信息生成文本的情感分类结果。典型的工作如Godbole\citeup{godbole2007large}利用英文词典构造了一个文本情感分析系统,并用该系统来对新闻文本的各种实体进行情感倾向分析。Balahur\citeup{balahur2011detecting}建立一个常识库用来记录各种引发情感的事件,从而识别出没有明确情感指示词汇出现的文本中的情感类别。其核心思想是把情景以本体的形式建模成一连串动作及与其相应的情感。Taboada\citeup{taboada2011lexicon}提出使用情感词典中情感词、程度副词和否定词特征对句子或短语进行情感分析。

近年来基于机器学习的情感分类研究得到快速发展。Abbasi\citeup{abbasi2008affect}在不同的数据集上构造出不同的特征,并采用更为强健的基于多种基分类器的集成方法,实现对文本的情感分析。Mohammad\citeup{mohammad2012portable}将情感词典与多元词频特征结合,使用支持向量机进行文本情感分析。张博\citeup{张博2015一种基于跨领域典型相关性分析的迁移学习方法}应用迁移学习技术将典型相关性分析引入情感分析中,基于特征映射迁移学习的思路,在保持各领域特有特征与领域共享特征相关性的基础上选择合适的基向量组合训练分类器,使降维后的相关特征在领域间具有相似的判别性,有效提高跨领域情感分类准确率。Gui\citeup{gui2015novel}应用类噪音估计算法,对跨语言迁移学习中负面样本进行检测和过滤,提高了在目标语言情感分析的性能。

随着深度学习在自然语言处理领域的应用,深度学习模型能利用神经网络获取词语、句子、段落和文档级别的分布式表示。在这种基于深度学习的分布式表示的前提下进行端到端的文本情感分析的方式已经成为现研究热点。Socher\citeup{socher2013recursive}利用递归神经张量网络模型对Stanford情感树库中标注了句子分析树和节点情感标签的句子进行复合表示学习。并应用该模型对输入句子进行逐个复合节点的情感判断,最终在根节点上得到整个句子的情感。 Kim\citeup{kim2014convolutional}将卷积神经网络进行部分改进后用于文本分析任务。实验结果证实了卷积神经网络在文本分析任务上的有效性。在这个的工作之后。Xu\citeup{xu2015show}提出了一种利用词向量表示构造平衡的训练数据集的方法。在依次构造词向量和句子向量表示后,使用SMOTE算法生成少数类别的样例,并最终构造出均衡的训练数据,有效提高了文本情感分类的性能。

文本情感分析在社交媒体应用主要集中在在线新闻评论的观点挖掘和微博(Twitter)文本情感分析。Potthast\citeup{potthast2012information}指出有关在线新闻评论的研究工作主要集中在信息检索方面,例如过滤,排序和评论摘要。对比于上述信息检索相关技术,在新评论的观点挖掘研究方向上探索相对较少。随着微博的风靡,与之相关的研究得到学术界和工商界的广泛关注。微博(Twitter)情感分析作为微博分析的重要基础任务,吸引了很多研究者的关注。以下分别介绍文本情感分析在两者的应用与研究。

对新闻评论的情感分析主要集中在极性检测和情感检测上。现有的研究大部分使用有监督的学习方法。Zhou\citeup{zhou2010research}对比了不同的特征在情感分析上的作用,Chardon探讨了使用话语结构预测新闻反应的作用。 Zhang\citeup{zhang2016ecnu}提出了一种用于标记情绪(如悲伤,惊喜和愤怒)的元分类器。在他们的方法中,作者在评论中使用了两个异构信息源:基于内容的信息和情感标签。Jakic提出了一种自动预测新闻反应中情绪极性的方法。在这项工作中,作者使用了领域基础知识和迁移学习,得到了Twitter数据迁移训练的分类器。moreo\citeup{moreo2012lexicon}提出了一种基于词法的方法,可以适应不同的领域。在他们的工作中,他们使用WordNet关系设计构建了一个结构化的词典。 Zhao\citeup{zhao2010jointly}提出一种利用评论的聚类对评论分类的无监督的方法,检索每一个评论的关键句子并提取其中的命名实体作为评论的目标。

从最近的研究情况来看,社交媒体文本的情感分析比起正式本文具有更大的挑战。它通常短且不正式,包含很多特殊的标记标签和表情符号和俚语,字母大小写也不一致。另一个问题是它倾向于不遵守语法规则,包含很多错误的拼写和很多单词的缩写。Kiritchenko\citeup{kiritchenko2014sentiment}提出根据Twitter文本短且非正式的特点,调整一系列表层特征提取方式,比如存在积极和消极的表情符号、标签、大写的字母和重复了字母的单词(比如sweeettt)。

\BiSubsection{文本立场分析研究现状}{Recommended figure format applied in \LaTeX}
上文论述了文本立场分析与文本情感分析有着本质的区别,文本立场分析更加关注文本反应出作者对于某一特定目标主题所持的立场和倾向。立场分析需要结合目标主题和情感信息,这比单独考虑文本的消息更加具有有挑战,对模型的建模能力也有更高的要求,现有立场分析的研究主要集中在国会辩论和网上辩论,这些领域的辩论作者会提出或者表现出清晰的立场倾向,语法和论述结构也相对固定化。但是对于一些用户主导的且表达方式更加随意内容,例如微博、Twitter、商品评论的立场分析的研究也相对较少。

现有立场分析的研究的主要基于有监督的学习,在Somasundaran\citeup{somasundaran2009recognizing}研究中,建立了论点触发词典,词典用来定位与抽取不同的论点,这些提取的论点、情绪表达、以及其目标作为立场分析分类器的特征。作者的实验表明单独用词频做特征比外加其他句法结构依赖性能效果并不会差很多,说明了在作者的任务上,其他特征对立场分析的影响相对较少。Hasan\citeup{hasan2013stance}在Anand使用的特征集的基础上,使用条件随机场(Conditional Random Fields,CRF)标注用户交互序列特征、整数线性规划(integer linear programming,ILP)建模作者意识形态约束特征,在 SVMs 分类器上的表现取得了最高 10\%的提升。通过两个支持立场语言的集合,Faulkner\citeup{faulkner2014automated}研究了文档级别在学生的文章中立场分析。Hasan\citeup{hasan2013stance}等提出了连续的评论之间不是完全独立的假设,连续的评论可以把问题定义成一个序列标注问题。Ahmed\citeup{ahmed2010staying}等提出一个改进的主题模型算法(Latent Dirichlet Allocation, LDA),作者把每一个词看成是立场倾向和主题的相互结合。Tutek使用随机森林(Random Forest, RF)、梯度提升决策树 (Gradient Boosting Decision Tree, GBDT)、 逻辑斯蒂回归 (Logistic Regression, LR)和支持向量机(Support Vector Machines, SVMs)四种基础的分类模型,构造多组语言学特征,作者利用了集成学习的思路,以F-measure作为集成学习的优化目标,线性组合四种基础模型的分类概率。其实验证明了基于多种基础分类的输出线性组合能明显的提高立场分类的效果。Sobhani的研究表明,一段文本表达的立场和文本的情感分析存在一定的关联 。若将文本的情感作为文本立场分析的特征时,能显著的提升文本立场分析的性能。Rajadesingan\citeup{rajadesingan2014identifying}研究用户级别的立场分析,作者提出了如果多个Twitter用户转发同一对有争议的话题,那这些用户很大可能拥有同意的立场。

在自然语言处理领域中,基于特征工程的方法能取得较好的效果,然而对文本提取特征词的特征工程需要大量的人力劳动和先验知识。而且基于传统的词向量的表达方式有语言缺乏关联,高纬特征稀疏特征表示,维度灾难等缺陷。Mikolov\citeup{mikolov2013efficient}提出了Word2Vec模型,解决了词向量训练速度慢,效率低的缺点。其利用了CBOW和Skip-Gram的两种语言模型,且创新性的提出了Hierarchical Softmax和负采样的词向量加速方法。为后续深度学习模型能在自然语言处理任务上打下了夯实的基础。有关文本立场分析的研究也开始关注能自动提取特征的深度学习。Wei\citeup{wei2016pkudblab}等利用多卷积核文本CNN的模型对Twitter文本进行有监督的立场分析,作者利用词窗口大小3,4,5的卷积提取文本中的特征,这算方法借鉴了自然语言处理中的N元组词(N-gram)的思想。此模型在Semeval2016英文立场分析数据集TaskA取得较好的成绩。Zarrella\citeup{zarrella2016mitre}使用迁移学习的方法,首先从大量的不标注Twitter数据中,选取了词频高于100的词汇,然后用Word2vec模型预训练好了每一个词的256维度的词向量,后面通过词向量相似度选取比较关键的以\#为前缀主题标签,通过训练神经网络预测主题标签。后通过迁移学习的思想对网络结构进行微调来达到立场分析的目的。实验结果表明,外部无标注数据的使用可以给有监督学习提供一些帮助,提高有监督学习的性能。

上述主要叙述了基于有监督的立场分析方法,虽然有监督的方式可以准确拟合训练集中,构造出对训练集有显著效果的特征, 但是标注大量有类标的训练集需要大量的人工成本,模型泛化能力也相对较弱,技术的应用场景也十分有限。为解决上述的缺点,研究人员开始展开对文本立场立场分析的无监督学习和弱监督学习的研究。 Johnson\citeup{johnson2016all}等通过基于不同方面的特征,构建6个局部弱监督基分类器。基于这些弱监督分类器在概率软逻辑(Probabilistic Soft Logic,PSL)算法的结合下,组成一个全局的弱监督分类模型。此模型在有关美国的32名政治人物的Twitter文本的立场分析任务中取得较好的性能。Augenstein\citeup{augenstein2016usfd}使用基于词袋的自动编码机(Auto-Encoder)学习文本的特征表示,并将学习得到的特征用于有监督的分类器中,解决了训练数据缺失的问题。




\BiSection{本文的主要研究内容和组织结构}{Recommended figure format applied in \LaTeX}

针对已有文本立场分析研究中忽略考虑主题目标信息的缺点,本文主要研究通过结合主题目标信息对社交媒体文本进行立场分析的方法,本文的工作主要从以下两个方面展开:

(1) 基于条件编码模型在社交媒体文本中立场分析,针对已有立场分析研究未引入主题目标信息的缺点,本文研究一种以条件编码的方式结合主题目标信息与文本信息的文本立场分析方法。用编码主题目标信息作为"先验知识"来指导立场分析中的文本信息的编码。结合文本立场分析语料中主题目标信息只包含少量信息,而文本包含绝大部分信息的特点,研究一种用单向与双向长短期记忆网络分别编码主题目标信息与文本信息的改进模型。设计了不引入主题目标信息、拼接主题目标信息与条件编码引入主题目标信息的对比实验,实验结果表明以条件编码的方式引入主题目标信息的方式能显著提高文本立场分析的效果。通过在SemEval2016英文立场分析数据集和NLPCC2016中文立场分析数据集与评测模型的对比,表明在改进后的条件编码模型在社交媒体文本立场分析具有较好的性能。

(2) 基于融合注意力机制与条件编码神经网络的社交媒体文本立场分析。基于主题目标对文本信息内容有着不同的侧重点为基础。本文将文本立场分析中主题目标信息作为注意力导向,给予文本信息不同权重的关注度。在不同关注度的文本信息中挖掘有关立场分析的模式。条件编码在“编码”阶段引入了主题目标信息,而注意力机制则在“解码”阶段引入主题目标信息。由于这两种方法以不同角度引入主题目标信息,本文研究一种基于融合注意力机制与条件编码神经网络的文本立场分析的方法。通过设计对比实验,表明条件编码与注意力机制在不同主题目标任务下各有优劣,融合后的模型具有显著的提升。融合模型在SemEval2016英文立场分析数据集上达到0.689的微平均F1值,在NLPCC2016中文立场分析数据集上达到了0.716的微平均F1值。对比两个数据集评测任务的最优系统分别提高了1.08\%和0.61\%。实验结果表明了融合注意力机制与条件编码模型在社交媒体文本立场分析的有效性。

本文的各章节的内容组织结构如下。

第1章为绪论,介绍了课题研究的相关背景与意义。其余各章主要内容如下:

第2章介绍立场分析领域的常见技术,分别介绍了文本情感分析的相关研究和文本立场分析的相关研究。

第3章介绍基于条件编码的社交文本立场分析。在此章首先介绍条件编码模型,然后研究用主题目标信息作为条件来编码文本信息的模型。在实验部分简介了词向量预训练、数据集、评测指标和数据预处理,后介绍实验结果与分析。

第4章介绍基于融合注意力机制与条件编码神经网络的社交媒体文本立场分析。由于注意力机制与条件编码是从两种不同角度引入主题目标信息,本章研究一种融合注意力机制与条件编码的文本立场分析方法,随后介绍了对该方法的验证实验和分析。

最后,结论回顾本文的主要研究内容,归纳并总结研究的不足支持和待解决的问题。

