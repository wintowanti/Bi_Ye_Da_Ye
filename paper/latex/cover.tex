% !Mode:: "TeX:UTF-8" 

\newcommand{\chinesethesistitle}{基于引入主题目标信息的社交媒体文本立场分析} %授权书用,无需断行
\newcommand{\englishthesistitle}{\uppercase{RESEARCH ON FUSION TOPIC INFORMATION BASED STANCE DETECTION IN SOCIAL MEDIA TEXT}} %\uppercase作用:将英文标题字母全部大写;
\newcommand{\chinesethesistime}{2017 年~12 月}  %封面底部的日期中文形式
\newcommand{\englishthesistime}{December, 2017}    %封面底部的日期英文形式

\ctitle{基于引入主题目标信息的社交媒体文本立场分析}  %封面用论文标题,自己可手动断行
\cdegree{\cxueke\cxuewei}
\csubject{计算机科学与技术}                 %(~按二级学科填写~)
\caffil{深圳研究生院} %(在校生填所在系名称,同等学力人员填工作单位)
\cauthor{颜瑶}
\csupervisor{徐睿峰教授} %导师名字
%\cassosupervisor{副导名}%若没有,请屏蔽掉此句。
%\ccosupervisor{联导名}%若没有,请屏蔽掉此句。


\cdate{\chinesethesistime}

\etitle{\englishthesistitle}
\edegree{\exuewei \ of \exueke}
\esubject{\emultiline[t]{Computer Science and Technology}}  %英文二级学科名
\eaffil{Shenzhen Graduate School}
\eauthor{Yao Yan}                   %作者姓名 (英文)
\esupervisor{Prof. Ruifeng Xu}       % 导师姓名 (英文)
%\eassosupervisor{Prof. Assosuper}%若没有,请屏蔽掉此句。
%\ecosupervisor{Prof. Cosuper}%若没有,请屏蔽掉此句。
\edate{\englishthesistime}

\natclassifiedindex{TM301.2}  %国内图书分类号
\internatclassifiedindex{62-5}  %国际图书分类号
\statesecrets{公开} %秘密

\iffalse
\BiAppendixChapter{摘~~~~要}{}
\fi
\cabstract{
摘要是论文内容的高度概括,应具有独立性和自含性,即不阅读论文的全文,就能获得必要的信息。
摘要应包括本论文的目的、主要研究内容、研究方法、创造性成果及其理论与实际意义。
摘要中不宜使用公式、化学结构式、图表和非公知公用的符号和术语,不标注引用文献编号。避免将摘要写成目录式的内容介绍。
}

\ckeywords{关键词~1;关键词~2;关键词~3;……;关键词~6(关键词总共~3~—~6~个,最后一个关键词后面没有标点符号)}

\eabstract{
Externally pressurized gas bearing has been widely used in the field of aviation, semiconductor, weave, and measurement apparatus because of its advantage of high accuracy, little friction, low heat distortion, long life-span, and no pollution. In this thesis, based on the domestic and overseas researching……

}

\ekeywords{keyword 1, keyword 2, keyword 3, ……, keyword 6 (no punctuation at the end) 英文摘要与中文摘要的内容应一致,在语法、用词上应准确无误。}

\makecover
\clearpage 
