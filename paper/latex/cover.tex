% !Mode:: "TeX:UTF-8" 

\newcommand{\chinesethesistitle}{结合主题目标信息的社交媒体文本立场分析} %授权书用,无需断行
\newcommand{\englishthesistitle}{\uppercase{RESEARCH ON STANCE DETECTION IN SOCIAL MEDIA TEXT By Incorporating TOPIC TARGET INFORMATION}} %\uppercase作用:将英文标题字母全部大写;
\newcommand{\chinesethesistime}{2017 年~12 月}  %封面底部的日期中文形式
\newcommand{\englishthesistime}{December, 2017}    %封面底部的日期英文形式

\ctitle{结合主题目标信息的社交媒体\\文本立场分析}  %封面用论文标题,自己可手动断行
\cdegree{\cxueke\cxuewei}
\csubject{计算机科学与技术}                 %(~按二级学科填写~)
\caffil{深圳研究生院} %(在校生填所在系名称,同等学力人员填工作单位)
\cauthor{颜瑶}
\csupervisor{徐睿峰教授} %导师名字
%\cassosupervisor{副导名}%若没有,请屏蔽掉此句。
%\ccosupervisor{联导名}%若没有,请屏蔽掉此句。


\cdate{\chinesethesistime}

\etitle{\englishthesistitle}
\edegree{\exuewei \ of \exueke}
\esubject{\emultiline[t]{Computer Science and Technology}}  %英文二级学科名
\eaffil{Shenzhen Graduate School}
\eauthor{Yao Yan}                   %作者姓名 (英文)
\esupervisor{Prof. Ruifeng Xu}       % 导师姓名 (英文)
%\eassosupervisor{Prof. Assosuper}%若没有,请屏蔽掉此句。
%\ecosupervisor{Prof. Cosuper}%若没有,请屏蔽掉此句。
\edate{\englishthesistime}

\natclassifiedindex{TP391.4}  %国内图书分类号
\internatclassifiedindex{004.9}  %国际图书分类号
\statesecrets{公开} %秘密

\iffalse
\BiAppendixChapter{摘~~~~要}{}
\fi
\cabstract{
随着互联网与移动互联网的飞速发展和多样化网络交流软件的普及使用,越来越多的网络用户可随时随地浏览热点新闻报道,并在网络中发表观点表达立场与情绪。现有的文本情感分析通常只对文本本身做出正负极性分析,无法深入挖掘文本作者对特定主题目标的立场倾向,而在很多应用场景下,关注更多的是文本立场倾向而不是文本情感。因此,针对特定主题目标的社交媒体文本立场分析研究具有巨大的研究价值与商业价值。

针对现有文本立场分析缺乏考虑主题目标信息的问题,本文研究了一种以条件编码的方式结合主题目标信息与文本信息的文本立场分析方法。用编码主题目标信息作为“先验知识”来指导立场分析中的文本信息的编码。根据文本立场分析语料的特点,改进了条件编码模型。改进后的模型在SemEval2016英文立场分析数据集和NLPCC2016中文立场分析数据集的微平均F1值分别为0.671与0.698。在不需要收集外部语料与手工设计特征的前提下,所提出基于条件编码方法具有较好的性能。实验表明以条件编码的方式引入主题目标信息能显著提高文本立场分析的性能。

针对主题目标对文本信息内容有着不同侧重点的特性,本文将文本立场分析中主题目标信息作为注意力机制的导向,给予文本信息不同权重的关注度,并在其中挖掘立场分析的模式。由于注意力机制与条件编码分别从“编码”与“解码”不同角度引入主题目标信息,本文提出了一种基于融合注意力机制与条件编码神经网络的文本立场分析的方法。融合模型在SemEval2016英文立场分析数据集和NLPCC2016中文立场分析数据的微平均F1值分别为0.689与0.716。对比两个数据集评测任务的最优系统,微平均F1值分别提高了1.08\%和0.61\%。显示了融合注意力机制与条件编码神经网络模型在社交媒体文本立场分析任务上的有效性。
}

\ckeywords{立场分析;深度学习;条件编码;注意力机制}

\eabstract{
With the rapid development of the Internet and mobile Internet and the popularity of  network communication software, people express their views and stance about hot social events and other topics at any time and in any place.  The existing text sentiment analysis only classify the texts into positive sentiment or negative sentiment, but can not get the stance of people on specific target. In many application scenarios, more attention is fouces on the textual stance rather than textual sentiment Therefore, the research of social media text stance detection towards specific target has specific target have great social and commercial value.

Aiming at the problem that the current stance detection research lacks consideration of the target information, this thesis proposes a stance detection method which combines the topic target information and the text information by conditional encoding. The encoding of the textual information in the stance detection is guided by the topic target information as "a priori knowledge." Combining with the stance detection of the characteristics of corpus,conditional coding model will be improved. The micro-average F1 values of the improved model are 0.671 and 0.698 in the SemEval2016 English dataset and the NLPCC2016 Chinese dataset, respectively. Without external collection of corpus and manual design features, the proposed conditional encoding model has good performance.it is shown that introducing the topic target information by conditional encoding can significantly improve the performance of the stance detection.

%Aiming at the lack of consideration of the target information in the existing stance detection research, this thesis studies a text stance detection method which combines the topic target information and the text information by conditional encoding. Through the design of comparative experiments, it is shown that the combining of  target information and text information by conditional encoding can significantly improve the performance of stance detection. Due to the characteristics of stance detection task, improving the conditional encoding model. In the case of SemEval2016 English dataset and NLPCC2016 Chinese dataset, The average micro F1 values ​​are 0.671 and 0.698, respectively. Conditional coding model performance is close to the best system for evaluation without the need of external corpus collection and manual design features.

%Based on the different target have different emphasis on the content of text information, this thesis will make target information as a focus of attention mechanism to give text information of different weights of attention, and then use weighed text information to extract the pattern of stance detection. Through  experiments, Conditional encoding and attention mechanism have their own advantages and disadvantages in different target. this thesis combine that two method into a model. The combined model get micro F1 values ​​of SemEval2016 dataset and NLPCC dataset were 0.689 and 0.716 respectively, which were respectively 1.08\% and 0.61\% higher than the top system in those of the two datasets. These results indicate that the proposed method is effective in stance detection.

Due to the characteristics that topic targets have different priorities for text content, the thesis proposes a method which make the topic target information as a guide mechanism of attention given different attention weights to text content, and in which extract the pattern of stance detection. Since attention mechanism and conditional eccoding introduce topic target information from different perspectives of "encoding" and "decoding", the thesis proposes a stance detection method based on neural network with attentional mechanism and conditional coding.the fusion model has a significant improvement. The micro-average F1 values of the fusion model are 0.689 and 0.716 in the SemEval English dataset and the NLPCC Chinese dataset , respectively. Comparing with the optimal system of two dataset evaluation tasks, the micro-average F1 value increases by 1.08\% and 0.61\% respectively. Experiments verify the effectiveness of the fusion attention mechanism and conditional encoding neural network model in the task of social media stance detection.

}

\ekeywords{stance detection, deeplearning, conditional encoding, attentional mechanisms}

\makecover
\clearpage 
